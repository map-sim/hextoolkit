\documentclass[11pt,a4paper]{article}
\usepackage[margin=1in,footskip=0.25in]{geometry}
\usepackage[dvipsnames]{xcolor}
\usepackage{amssymb,amsmath}
\usepackage[utf8]{inputenc}
\usepackage[OT4]{fontenc}
\usepackage{graphicx}
\usepackage{multicol}

\newcommand{\elektrownia}[2]{
  \begin{picture}(20, 20)(#1)
    \put(15,15){\color{blue}\circle{13}}    
    \put(12.5,12.5){\color{blue}\circle{5}}    
    \put(17.5,12.5){\color{blue}\circle{5}}    
    \put(17.5,17.5){\color{blue}\circle{5}}    
    \put(12.5,17.5){\color{blue}\circle{5}}    
    \put(-7,-2){\scriptsize \color{blue}#2}    
  \end{picture}  
}
\newcommand{\kopalnia}[2]{
  \begin{picture}(20, 20)(#1)
    \put(15,15){\color{blue}\circle{13}}
    \put(15,15){\color{blue}\circle*{4}}
    \put(-1.5,-2){\scriptsize \color{blue}#2}    
  \end{picture}
}
\newcommand{\magazyn}[2]{
  \begin{picture}(20, 20)(#1)
    \put(15,20){\color{blue}\line(1, 0){10}}
    \put(15,15){\color{blue}\line(1, 0){10}}
    \put(15,25){\color{blue}\line(1, 0){10}}
    \put(25,15){\color{blue}\line(0, 1){10}}
    \put(20,15){\color{blue}\line(0, 1){10}}
    \put(15,15){\color{blue}\line(0, 1){10}}
    \put(4,3){\scriptsize \color{blue}#2}    
  \end{picture}
}
\newcommand{\mikser}[2]{
  \begin{picture}(20, 20)(#1)
    \put(15,15){\color{blue}\circle{13}}
    \put(10,10){\color{blue}\line(1, 1){10}}
    \put(10,20){\color{blue}\line(1, -1){10}}
    \put(2,-2){\scriptsize \color{blue}#2}    
  \end{picture}
}
\newcommand{\wartownia}[2]{
  \begin{picture}(20, 20)(#1)
    \put(21,3){\color{blue}\circle{8}}
    \put(4,-13){\scriptsize \color{blue}#2}    
  \end{picture}
}

\newcommand{\laboratorium}[2]{
  \begin{picture}(20, 20)(#1)
    \put(25,10){\color{blue}\circle{7}}
    \put(15,10){\color{blue}\line(1, 1){10}}
    \put(15,10){\color{blue}\line(1, -1){10}}
    \put(35,10){\color{blue}\line(-1, 1){10}}
    \put(35,10){\color{blue}\line(-1, -1){10}}
    \put(20.5,-8){\scriptsize \color{blue}#2}    
  \end{picture}
}

\newcommand{\transmiter}[2]{
  \begin{picture}(20, 20)(#1)
    \put(25,5){\color{blue}\circle{13}}
    \put(22,0){\color{blue}\rotatebox{90}{$\gg$}}
    \put(4,-13){\scriptsize \color{blue}#2}    
  \end{picture}
}
\newcommand{\deweloper}[2]{
  \begin{picture}(20, 20)(#1)
    \put(25,5){\color{blue}\circle{13}}
    \put(21.5,10){\color{blue}\rotatebox{-90}{$\gg$}}
    \put(6,-13){\scriptsize \color{blue}#2}    
  \end{picture}
}
\newcommand{\wyrzutnia}[2]{
  \begin{picture}(20, 20)(#1)
    \put(15,15){\color{blue}\circle{13}}
    \put(15,6){\color{blue}\line(0, 1){18}}
    \put(6,15){\color{blue}\line(1, 0){18}}
    \put(-3,-3){\scriptsize \color{blue}#2}    
  \end{picture}
}

\begin{document}
\title{Draft: Interplanetary Exploatation Companies}
\author{Krzysztof Czarnecki}
\date{Sierpień, 2023}
\maketitle

\section{Wstęp}

Koniec XXV wieku to okres szybkiego rozwoju górnictwa kosmicznego. Ludzkość intensywnie eksploatuje pobliskie ciała niebieskie -- planety, ich naturalne satelity oraz planetoidy, w tym głównie Księzyc, Mars, Wenus, oraz księżyce Jowisza i Saturna. Ważnym źródłem surowców, są także komety przelatujące w poblizu Ziemii. Tu jednak okno czasowe pozwalające na operowanie na ich powieszchni jest ograniczone do kilku miesięcy ze względu na relatywnie krótki czas przebywania w pobliżu Słońca, które zapewnia akceptowaną temperaturę dla pracy infrastruktury wydobywczej. Sektor górnictwa kosmicznego zmonopolizowały prywatne przedsiębiorstwa głównie ze Stanów Zjednoczonych, Chin, Europy, Indii i Japonii, ale znana jest również znaczna liczba międzynarodowych konsorcjów tworzonych przez firmy ze wszystkich kontynentów świata (poza Antarktydą). Te prywatne korporacje i konsorcja zaczeły organizować się na wzór kolonialnych kopanii handlowych znanych z XVII, XVIII i XIX wieku. Mają swoje własne unikatowe technologie, swoje kosmiczne oddziały militarne, oraz wysoką autonomie od państw macierzystych z racji działanlości poza juryzdykcją prawną tych państw.  

Górnictwo kosmiczne zapewnia ogromne zyski, ale jest też przedsięwzięciem szalenie niebezpiecznym i ze względu na gigantyczne odległości, trudnym do kontroli z Ziemii. To powoduje pokusę, aby korporacje kosmiczne stosowały wroge oddziaływania miltarne wobec konkurentów działających na tych samych złożach surowców. Włączając w to dywersję oraz fizyczną eliminację persolelu i infrastruktury wraz z jej siłowym przejmowaniem. Aby, ograniczyć ryzyko takich działań, w roku 2499 wszystkie liczące się w branży kosmicznej państwa na Ziemii podpisały tzw. \textbf{Memorandum Pekińskie}, które zabrania wynoszenia do przestrzeni kosmicznej bronii ofensywanej rozumianej jako mobilne systemy rażenia. Sygnatariusze tego porozumienia mają obowiązek kontroli działających na ich terytorium korporacji i powinny uniemożliwić wynoszenia takich środków walki do przestrzeni kosmicznej. W rezultacie głównymi systemami militarnymi występującymi na stacjach wydobywczych stały się stacjonarne wyrzutnie rakiet w zamyśle przeznaczone tylko do obrony. Czy jednak dowódcy i załogi tych stacjonarnych jednostek będą je wykorzystywać zgodnie z przeznaczeniem? Na to pytanie odpowiedzą zarządy korporacji kosmicznych i czas...
\newpage

\section{Surowce i Processy}

Gracze mogą wydobywać surowce proste, oznaczane pojedynczymi literami alfabetu: $A$, $B$ i $C$, dzięki kopalniom wybudowanym na złożach. Złoża charakteryzują się prawdopodobieństwem wydobycia pojedynczego surowca danego typu -- test wydobycia przeprowadza się kostką K6. Jedno złoże może być zdolne do dostarczenia więcej niż jednego rodzaju surowca. Dokładnej ta procedura jest opisana przy okazji omówienia objektu kopalni. Te proste surowce są dalej wykorzystywane w mikserach jako substraty do produkcji towarów złożonych oznaczonych związkami liter: $AB$, $BC$, oraz $AC$. A zatem kompanie kosmiczne przetwarzają surowce już na powieszchni ciał niebieskich, na których operują. Finalnie, surowce złożone są wykorzystywane jako substraty w tzw. efektorach do produkcji konkretnych środków przydatnych w grze (z określonym prawdopodobieństwem sukcesu wytworzenia):
\begin{itemize}
  \setlength{\parskip}{0pt}
  \setlength{\itemsep}{0pt plus 1pt}
\item transmisja towarów na ziemie (punkty zwycięstwa),
\item budowanie nowych budynków lub odbudowywanie zniszczonych modułów,
\item wytwarzanie rakiet dla wyrzutni przeznaczonych do niszczenia innych budynków,
\item lub odkrywanie zaawansowanych technologii.
\end{itemize}

\rule{-6mm}{0mm}\elektrownia{4,12}{elektrownia} $\to\to \Big($ \kopalnia{4,12}{kopalnia} $\to$ \magazyn{8,17}{magazyn} $\to$ \mikser{6,12}{mikser} $\to$ \magazyn{8,17}{magazyn} $\to\big($\rule{-2mm}{0mm}\laboratorium{14,7}{lab} $||$ \transmiter{14,2}{transmiter} $||$ \rule{3mm}{0mm}\deweloper{14,2}{konstruktor}\rule{3mm}{0mm} $||$ \wyrzutnia{3,12}{wyrzutnia} $\big)\Big) \rule{1mm}{0mm}||$ \rule{-4mm}{0mm}\wartownia{8,2}{wartownia}

\rule{0cm}{1cm}Diagram dozwolonego przesuwania surowców jest zaprezentowany powyżej. $\to\to$ oznacza bycie zasilanym przez elektrownie (dostępne jest jedno źródło zasilania -- reaktor nuklearny). $\to$ oznacza normalne przemieszczenie surowca. $||$ oznacza alternatywne budynki $/$ efektory.

\section{Elementy Mapy}

Mapa składa się z 2 rodzajów elementów: powieszchnie terenowe i objekty infrastruktury. Objekty infrastruktury należą do konkretnego gracza, który ma nad nimi pełną kontrolę. Budynki mogą realizować swoje podstawowe funkcje opisane w następnej sekcji (jeżeli są zasilane). Gracz może zniszczyć każdy swój budynek nie tracąc żadnego punktu zwycięstwa na początku tury (oznacza to wysadzenie go i ewakuacje personelu oraz uratowanie cennych komponentów modułu; nie przynosi to żadnych bezpośrednich korzyści graczowi, ale może go uratować od straty punktów zwycięstwa -- o czym będzie dalej, lub być wykorzystane do negocjacji w ramach umów pomiędzy graczami). Należy przy tym pamiętać, że bilateralne umowy ustne między graczami nie są nigdy wiążące.

Drugim rodzajem elementów mapy są tereny. Na mapie, można znaleźć m.in. zakreślone obszary, w których znajduja się złoża surowców prostych. Właściwością takich terenów są nazwy surowców, które można tu wydobywać. Inną właściwością terenu jest możliwść lub brak możliwości budowania objektów (np. brak możliwości budowy na terenach płynnych, stromych lub silnie toksycznych). Kolejną możliwością, są strefy aktywne sejsmicznie lub niestabilne z innych powodów, które z określonym prawdopodobieństwem powodują uszkodzenia modułów budynków. Dokładniejsze opisy poszczególnych właściwości terenów powinny być dostarczone wraz ze scenaruszem rozgrywki.

Dodatkowym elementem gry jest tzw. \underline{trudność wynoszenia ładunków na orbitę} (zawsze w przedziale od 1 do 12). Trudność ta wpływa na działanie tramsmitera (efektora przynoszącego punkty zwycięstwa w rozgrywce) -- jej wpływ jest zdefiniowany przy opisie budynku. Opis wyznaczania trudności warunków wynoszenia na orbitę powinien być zawarty w każdym scenariuszu -- reguluje on blans gry i zmienia opłacalność różnych strategii.

\newpage

\subsection{Sposób Losowego Rozmieszczenia Złóż}

\begin{picture}(200, 200)
  \put(10, 10){ \framebox(180,180){}}
  \put(40, 150){I}
  \put(20, 150){\qbezier(0, 0)(40, 40)(80, -20)}
  \put(20, 150){\qbezier(0, 0)(40, -40)(80, -20)}
  \put(50, 80){II}
  \put(30, 80){\qbezier(0, 0)(40, 40)(80, 20)}
  \put(30, 80){\qbezier(0, 0)(40, -40)(80, 20)}
  \put(120, 60){III}
  \put(110, 80){\qbezier(0, -60)(40, 0)(80, 20)}
  \put(110, 80){\qbezier(0, -60)(-40, 0)(80, 20)}
  \put(190, 95){
    \begin{minipage}{8.2cm}
      \begin{enumerate}
        \setlength{\parskip}{0pt}
        \setlength{\itemsep}{0pt plus 1pt}
      \item Na poczatku na mapie są naszkicowane obszary oznaczone liczbami rzymskimi
      \item Należy zdefiniować listę$/$kombinację złóż prostych do rozmieszczenia na mapie, np.:
        \begin{itemize}
          \setlength{\parskip}{0pt}
          \setlength{\itemsep}{0pt plus 1pt}
        \item $A$
        \item $1/2 B$
        \item $1/2 B$, $1/3 C$
        \end{itemize}
      \item Należy losowo przyporządkować złoża do każdego obszaru (I, II, lub III) -- gracze nie znają tego przyporządkowania do chwili budowy pierwszej kopalni.
      \item Przyporządkowanie jednego tylko złoża do terenu następuje w chwili, gdy któryś gracz zbuduje pierwszą kopalnie na tym terenie.
      \end{enumerate}
    \end{minipage}
  }
\end{picture}

\section{Objekty Infrastruktury}

W poniższych tabelach przedstawiono budynki, które mogą występować w grze. Pierwsze 2 kolumny oznaczają nzwę budynku w języku polskim oraz angielskim. Trzecia kolumna definiuje z ilu modułów składa się każdy budynek -- moduły mozna interpretować jako liczbę trafień możliwych do odebrania przez budynek w wyniku ostrzału rakietowego. W czwartej kolumnie naszkicowano reprezentacje graficzne budynków umieszczane na mapie. Kolejna kolumna o tytule zasięg określa jaka jest maksymalna odległosc oddziaływania danego budynku. Kolumna o tytule interwał definiuje odległość, w jakiej nie może powstać żaden inny budynek (w praktyce jest to zawsze większy interwał spośród 2 rozpatrywanych budynków). W przypadku efektorów wprowadzona jest dodatkowa kolumna o nazwie substraty, która określa jakie surowce są akceptowane przez dany budynek dla zapewnienia jego nominalnego działania.

Odległość w zasięgu bazowym definiuje się jako mniejszą lub równą wartości umieszczonej w kolumnie zasięg w tabelach budynków. Zasięg rozszerzony oznacza odległość większą niż zasięg bazowy, ale maksymalnie podwojoną wartość zasięgu z tabel. Zasięg rozszerzony mogą wykorzystywać tylko efektory.

\subsection{Budynki Produkcyjne}
\begin{center}
\begin{tabular}{| c | c | c | c | c | c |}
  \hline
  \textbf{nazwa PL} & \textbf{nazwa EN} & \textbf{moduły} & 
  \textbf{rysunek} & \textbf{interwał}\\
  \hline
  kopalnia & mine & 2 & \kopalnia{7,8}{} & 3 cm\\  
  \hline
  mikser & mixer & 2 & \mikser{6,8}{} & 3 cm\\
  \hline
\end{tabular}
\end{center}

\subsection{Budynki Wspierające}
\begin{center}
\begin{tabular}{| c | c | c | c | c | c |}
  \hline
  \textbf{nazwa PL} & \textbf{nazwa EN} & \textbf{moduły} & 
  \textbf{rysunek} & \textbf{zasięg} & \textbf{interwał}\\
  \hline
  reaktor nuklearny & nuclear reactor & 8 & \elektrownia{6,8}{} & $\infty$ & 6 cm \\  
  \hline
  magazyn & storage & 2 & \magazyn{12,13}{} & 6 cm & 3 cm\\
  \hline
  wartownia & observation post & 1 & \wartownia{13,-2}{} & -- & 2 cm \\
  \hline
\end{tabular}
\end{center}

\subsection{Budynki -- Efektory}
\begin{center}
  \begin{tabular}{| c | c | c | c | c | c | c |}
    \hline
    \textbf{nazwa PL} & \textbf{nazwa EN} & \textbf{moduły} & \textbf{rysunek} &
    \textbf{zasięg} & \textbf{interwał} & \textbf{substraty} \\
    \hline
    laboratorium & laboratory & 2 & \laboratorium{14,2}{} & -- & 3 cm & $BC+AC$\\
    \hline
    konstruktor & developer & 3 & \deweloper{14,-2}{} & 8 cm & 3 cm & $AB$\\    
    \hline
    transmiter & transsmiter & 2 & \transmiter{14,-2}{} & -- & 4 cm & $BC$\\    
    \hline
    wyrzutnia rakietowa & launcher & 3 & \wyrzutnia{5,8}{} & 30 cm & 4 cm & $AC$\\
    \hline
  \end{tabular}
\end{center}



\subsection{Opis Wszystkich Budynków}

\subsubsection{Reaktor Nuklearny $/$ Nuclear Reactor}

Pojedynczy reaktor nuklearny jest małą modułową elektrownią atomową i jest strategicznie ważny w rozgrywce. Każda tego typu elektrownia może zasilać jednocześnie $X$ budynków (sam reaktor atomowy jest zasilany automatycznie, a zatem nie wchodzi w skład tej liczby). W przypadku przekroczenia wspomnianej liczby na skutek, czy to zniszczenia reaktora, czy zbudowania zbyt wielu budynków gracz musi wybrać, które budynki będą zasilane a które nie. Niezasilane budynki pozostają nieaktywne do chwili powtórnego przywrócenia zasilania -- można zmieniać konfigurację zasilanych budynków w każdej turze rozgrywki. Reaktor atomowy zaczyna produkować energię elektryczną dopiero wtedy, gdy zbudowano wszystkie jego moduły. Działający reaktor przestaje zasilać dopiero wtedy, gdy zostały zniszczone wszystkie jego moduły. Wystarczy choćby jeden sprawny moduł, aby móć zasilać nominalną liczbę budynków~$X$.

Zniszczenie wszystkich reaktorów atomowych danego gracza, znacząco ogranicza możliwości zasilania jego infrastruktury. Należy jednak pamiętać, że każdy moduł ma swój autonomiczny system podtrzymywania napięcia elektrycznego za pomocą paneli słonecznych, oraz awaryjnych akumulatorów. Te elementy pozwalają na przetrwanie załogi, ale nie są wstanie zapewnić operacyjnego działania tego budynku. Jednak w trybie alarmowym gracz może wygospodarować w każdym module trochę energi i skoncentrować ją w wybranych budynkach -- oznacza to możliwość aktywowania tylko jednego budynku na każde $X$ niezniszczonych modułów posiadanych przez danego gracza.

Każdy gracz może wspomagać dowolnego innego gracza w zasilaniu jego infrastruktury pod warunkiem, że któryś z nich odkrył technologię na to pozwalającą -- w takim wariancie gracz może dalej zasilać przynajmiej cześć swojej infrastruktury. Należy jednak pamietać, że zasilanie budynków innego gracza powoduje zmiejszenie możliwości zasilania swoich własnych budynków. Gracz udzielający pomocy decyduje ile budynków otrzymuje zasilanie, natomiast to właściciel infrastruktury zawsze wybiera, które budynki są zasilane, a które nie. Udostępnianie zasilania ma sens wtedy, gdy w rozgrywce uczestniczy większa liczba graczy, np. aby zachować balans gry lub w grach zespołowych np. 2:2. Reaktor nuklearny powinien znajdować się w startowym zestawie budynków, chyba, że rozgrywany scenariusz mówi inaczej.

\subsubsection{Kopalnia $/$ Mine }

Kopalnie wydobywają podstawowe surowce ze skorupy eksloatowanego ciała niebieskiego. Kopalnia składa się z 2 modułów i w podstawowej konfiguracji wydobywa tylko jeden surowiec na turę z prawdopodobieństwem określonym przez własciwości złoża (chyba że gracz posiada technologię rozszerzającą ten limit). Wydobyty surowiec musi natychmiast być umieszczony w sąsiadującym magazynie (przy uwzględnieniu limitu tego magazynu). Jeżeli żaden magazyn nie może przyjąć surowców, to nie mogą one być wydobyte. Jeżeli kopalnia nie jest zasilana to nie może wydobywać surowców.

\subsubsection{Magazyn $/$ Storage}

Magazyny są przeznaczone do przechowywania surowców -- maksymalnie 4 (chyba że gracz posiada technologię rozszerzającą ten limit), oraz przesyłania surowców pomiędzy budynkami. Każdy magazyn może odebrać w jednej turze maksymalnie 2 surowce z kopalń lub mikserów (chyba że gracz posiada technologię rozszerzającą ten limit). Pomiędzy magazynami nie ma limitu przesyłanych surowców (wyjaśnieniem może tu być paletyzacja na wejściu do magazynu, potem transport jest dużo łatwiejszy). W praktyce surowce można dowolnie przemieścić w ramach danej sieci magazynów. Każdy magazyn w podstawowej konfiguracji składa się z 2 modułów. Nie zasilany magazyn nie może odbierać surowców ani wydawać ich innym magazynom i mikserom. Jednak może je w dalszym ciągu wydawać efektorom -- gdyż to efektory posiadają środki do ich transportu. Magazyn nie posiada limitu na wydawanie surowców. Jeżeli magazyn traci ostatni moduł to gracz musi zniszczyć wszystkie surowce w nim przetrzymywane. Transport surowców pomiędzy budynkami zawsze może odbywać się tylko w zasięgu przypisanym magazynom (również do efektorów). Gracz może wysyłać surowce zgromadzone w swoich magazynach do innych graczy pod warunkiem spełnienia normalnych zasad przesyłu (w tym do efektorów obcego gracza). Wysłany towar nie może jednak być transportowany dalej w ramach tej tury. Oboje gracze uczestniczący w tranzakcji muszą się zgodzić na taki transport. Można w ten sposób dokonywać tranzakcji dwukierunkowych. Przesunięcia surowców należy dokonywać w kolejności rozgrywki -- przy czym gracze nie muszą dotrzymywać ustnych obietnic.

\subsubsection{Mikser $/$ Mixer}

Miksery mogą produkować dowolne surowce złożone wykorzystując do tego celu surowce proste jako substraty. Każdy mikser w podstawowej konfiguracji może wyprodukować tylko jeden surowiec złożony w danej turze (chyba, że gracz posiada technologię rozszerzającą ten limit). Taki wyprodukowany surowiec musi natychmiast być umieszczony w sąsiadującym magazynie. Ten surowiec wchodzi w sumę limitu tego magazynu surowców, których jest on wstanie przyjąć w danej turze z obiektów nie będących magazynami. Jeżeli mikser nie posiada w sąsiedztwie wymaganych substratów (w magazynach będących w zasięgu) albo nie może umieścić wyprodukowanego surowca w żadnym magazynie, to produkcja nie dochodzi do skutku. Każdy mikser składa się z 2 modułów w podstawowej konfiguracji.

\subsubsection{Wartownia $/$ Observation Post}

Wartownie to małe budynki, które mogą udostepniać pole widzenia wyrzutniom rakietowym, jak również dawać zasięg dla akcji sabotujących -- przy czym gracz musi posiadać poszczególne technologie dla tych akcji. Jeżeli gracz nie dysponuje odpowiednimi zdolnościami dywersyjnymi lub centralnym systemem dowodzenia to zyski z posiadania wartowni są umiarkowane -- blokowanie terenu przed zabudową. Budynek wartowni bo wybudowaniu pierwszego i jedynego modułu standardowego automatycznie dostaje pancerz pasywny (nawet, jeżeli gracz nie posiada tej technologi). Wartownia nie wymaga zasilania, ale jej pojedynczy moduł wlicza się do modułów generujących zasilanie alarmowe. Zniszczenie lub przejęcie wartowni przez innego gracza nie powoduje utraty punktu zwycięztwa przez właściciela~wartowni.

\subsubsection{Konstruktor $/$ Developer}

Konstruktor służy do budowania i odbudowywania budynków w jego zasięgu (lub recyklinku, jeżeli gracz posiada odpowiednią technologię). Składa się on m.in. z drukarek 3D, które drukują gabarytowe elementy potrzebne do budowy. Zakłada się, że małe wyspecjalizowane podzespoły, jak elektronika i komponenty precyzyjne nie możliwe do wytworzenia na powieszchni są dostarczane na bieżąco z orbity. Każdorazowo w sytuacji kolizji o tym, który gracz może zbudować decyduje kolejność graczy. Budowa każdego modułu budynku zajmuje jedną turę (dopuszczalne jest, że 2 i więcej konstruktorów buduje jeden budynek). Przy pierwszym budowanym module gracz musi zadeklarowć jakiego typu budynek buduje. Budynek staje sie operacyjny dopiero po zbudowaniu wszystkich wymaganych modułów budynku. Udana budowa każdego modułu kosztuje 2 surowce $AB$. W przypadku nieudanej próby zbudowania modułu, gracz musi ponieść koszt pojedynczego surowca $AB$. Nieudane próby oznaczają wypadki budowlane, o które bardzo łatwo na nieprzyjaznych obcych ciałach nebieskich. Za każdym razem wymagane surowce $AB$ muszą się znajdować w sąsiadujących magazynach. Konstruktor nie może dokonywać auto-napraw. Prawdopodobieństwo udanej budowy modułu zależy od uszkodzeń konstruktora i odległości do budowanego budynku według poniższej tabeli:
\begin{center}
  \begin{tabular}{| c | c | c |}
    \hline
     & \textbf{liczba nieuszko-} & \textbf{prawdopodobieństwo}\\
    \textbf{Zasięg} & \textbf{dzonych modułów} & \textbf{1 budowy modułu}\\
    \hline
     bazowy & 3 & 1 \\
    \hline
    -,,-  & 2 & $2/3$ \\
    \hline
    -,,- & 1 & $1/3$ \\
    \hline
    rozszerzony & 3 & $5/6$ \\
    \hline
    -,,- & 2 & $1/2$ \\
    \hline
    -,,- & 1 & $1/6$ \\
    \hline
  \end{tabular}
\end{center}

\subsubsection{Wyrzutnia Rakietowa $/$ Launcher}

Podstawową rolą wyrzutni rakietowej jest rażenie wrogich budynków na dystansie. Wyrzutnia w bazowej konfiguracji składa się z 3 modułów i może wystrzeliwać 2 pociski na turę. Jeżeli pierwszy strzał chybił wyrzutnia może podjąć kolejną próbę (lub kolejne próby, jeżeli inne zasady umożliwiają większą liczbę). Prawdopodobieństwo trafienia zależy od odległości do celu i pola widzenia. Pole widzenia budynków zależy od rozkładu obszarów terenowych i powinno być zdefiniowane w każdym scenariuszu (gracz może wykorzystać pole widzenia innego budynku pod warunkiem posiadania odpowiedniej technologi). Koszt każdego pocisku to jeden surowiec $AC$, który musi być dostępny w sąsiadującym magazynie. Wyrzutnia rakietowa jest operacyjna już po zbudowaniu jej pierwszego modułu (jest to wyjątek). Zakłada się, że wszystkie wyrzutnie strzelają jednocześnie w ramach każdej próby (gracze deklarują cele w standardowej kolejności). Dobór celów i ich kolejności należy przeprowadzić we wszystkich wyrzutniach przed fazą ostrzału. Jeżeli wszystkie moduły jakiegokolwiek budynku zostaną zniszczone wtedy uznaje się, że cały budynek uległ zniszczeniu. Zawsze utrata jednego budynku przez gracza (z wyjątkiem wartowni oraz smodzielnym zniszczeniu go) oznacza utratę przez niego punktu zwycięstwa. Dodatkowo wyrzutnia rakietowa może być ulepszona dzięki kilku zaawansowanych technologii. Prawdopodobieństwo zniszczenia jednego modułu wrogiego budynku określa poniższa tabela:
\begin{center}
\begin{tabular}{| c | c | c | c |}
  \hline
   \textbf{zasięg} & \textbf{pole widzenia} & \textbf{1. próba} & \textbf{2 próba} (i kolejne) \\
  \hline
  bazowy & tak   & $2/3$ & $5/6$ (korekta operatorów) \\  
  \hline
  -,,- & nie   & $2/3$ & $2/3$ \\  
  \hline
  rozszerzony & tak & $1/3$ & $1/2$ (korekta operatorów) \\  
  \hline
  -,,- & nie  & $1/3$ & $1/3$ \\  
  \hline
  rozszerzony  &  &  &  \\  
  + \ref{gps-tech}. technologia & tak  & $1/2$ & $2/3$ (korekta operatorów) \\  
  \hline
  -,,- & nie   & $1/2$ & $1/2$ \\  
  \hline
\end{tabular}
\end{center}

\subsubsection{Laboratorium $/$ Laboratory}

Laboratoria prowadzą research, co symbolizuje wytwarzanie punktów nauki z pary surowców złożonych: $BC + AC$ (co jest równoważne jednemu punktowi nauki). Gracz na końcu tury musi skrycie wybrać technologie zaawansowaną, jeżeli uzbierał wystarczającą liczbę punktów nauki. Gracz nie musi okazywać jaka technologię wybrał do chwili jej pierwszego zastosowania. Koszt każdej kolejnej technologii rośnie liniowo i określa go tabela w sekcji 5. Każde laboratorium w podstawowej wersji składa się z 2 modułów. 

\subsubsection{Transmiter $/$ Transsmiter}

Każdy transmiter może podjać jedną probę (chyba że gracz posiada technologię rozszerzającą ten limit) wytransferowania surowca $BC$ na orbitę, a potem na Ziemię -- jest to sposób na zdobywanie punktów zwycięstwa w grze. Nieudana próba transferu oznacza utratę surowca, który miał być wytransferowany. Prawdopodobieństwo sukcesu zależy od aktualnych warunków wynoszenia towarów na orbitę w następujący sposób (gracze mogą dysponować technologiami zmieniającymi te warunki):
\begin{center}
  \begin{tabular}{| c | c |}
    \hline
    \textbf{trudność wynoszenia ładunków na orbitę} & \textbf{prawdopodobieństwo transferu}\\
    \hline
     $\ge$ 8 & 0 \\
    \hline
    7 & $1/6$ \\
    \hline
    6 & $1/3$ \\
    \hline
    5 & $1/2$ \\
    \hline
    4 & $2/3$ \\
    \hline
    3 & $5/6$ \\
    \hline
     $\le$ 2  & 1 \\
    \hline
  \end{tabular}
\end{center}

%% \subsection{Przykład Mapy (+ Przepływy Surowców)}
%% 
%% \begin{center}
%% \begin{picture}(240, 170)
%%   %\put(0, 0){ \framebox(240,170){}}
%%   \put(20, 110){\elektrownia{7,8}{}}
%%   \put(20, 150){\textcolor{gray}{
%%     \qbezier(0, 0)(40, 40)(80, -20)
%%     \qbezier(0, 0)(40, 0)(80, -20)
%%     \put(10,1){A}
%%   }}
%%   \put(80, 130){\kopalnia{7,8}{}}
%%   \put(70, 95){\magazyn{7,8}{}}
%%   \put(120, 90){\mikser{7,8}{}}
%%   \put(150, 105){\magazyn{7,8}{}}
%%   \put(180, 135){\deweloper{7,8}{}}
%% 
%%   \put(90, 130){\vector(0,-1){17}{}}
%%   \put(94, 107){\vector(3,-1){29}{}}
%%   \put(69, 76){\vector(1,0){31}{}}
%%   \put(110, 78){\vector(1,1){15}{}}
%%   \put(140, 102){\vector(2,1){19}{}}
%%   \put(85, 101){\vector(1,-1){21}{}}
%%   \put(174, 120){\vector(2,1){19}{}}
%%   
%%   \put(10, 80){\textcolor{gray}{
%%     \qbezier(0, 0)(40, 40)(70, -20)
%%     \qbezier(0, 0)(20, -10)(70, -20)
%%     \put(10,0){B}
%%   }}
%%   \put(50, 70){\kopalnia{7,8}{}}
%%   \put(64, 56){\kopalnia{7,8}{}}
%%   \put(90, 60){\magazyn{7,8}{}}
%%   \put(105, 45){\magazyn{7,8}{}}
%%   \put(140, 70){\mikser{7,8}{}}
%%   \put(170, 80){\magazyn{7,8}{}}
%% 
%%   \put(90, 38){\vector(3,2){25}{}}
%%   \put(128, 62){\vector(3,2){17}{}}
%%   \put(114, 76){\vector(1,0){30}{}}
%%   \put(159, 80){\vector(2,1){21}{}}
%%   \put(193, 92){\vector(1,0){20}{}}
%%   \put(209, 88){\wyrzutnia{7,8}{}}
%% 
%%   \put(69, 20){\vector(1,1){11}{}}
%%   \put(77, 55){\vector(1,-2){8}{}}
%%   \put(93, 36){\vector(1,0){31}{}}
%%   \put(93, 33){\vector(1,0){32}{}}
%%   \put(140, 38){\vector(1,0){30}{}}
%%   \put(184, 38){\vector(1,0){30}{}}
%% 
%%   \put(10, 30){\textcolor{gray}{
%%     \qbezier(0, 0)(40, 40)(60, -20)
%%     \qbezier(0, 0)(20, -10)(60, -20)
%%     \put(20,0){C}
%%   }}
%%   \put(50, 10){\kopalnia{7,8}{}}
%%   \put(70, 20){\magazyn{7,8}{}}
%%   \put(120, 30){\mikser{7,8}{}}
%%   \put(160, 25){\magazyn{7,8}{}}
%%   \put(200, 42){\transmiter{7,8}{}}
%%   \put(190, 62){\laboratorium{7,8}{}}
%% 
%%   \put(183, 41){\vector(1,1){19}{}}
%%   \put(186, 85){\vector(1,-1){17}{}}
%% 
%%   \put(155, 85){$AC$}
%%   \put(135, 107){$AB$}
%%   \put(146, 39){$BC$}
%%   \put(125, 77){$A$}
%%   \put(110, 85){$B$}
%%   \put(97, 47){$C$}
%%   \put(105, 37){$B$}
%%   \put(105, 24){$C$}
%% \end{picture}
%% \end{center}


% \section{\textcolor{orange}{Wojna Minowa (dodatek; inspiracja wojną UA-RUS'23)}}
% 
% \textcolor{orange}{Mapa jest pozielona na kwadraty o wymiarach 2 cm $\times$ 2 cm. Szachownica ta służy głównie do określenia obszarów na cele wojny minowej. Każdy gracz w początkowej fazie tury może umieścić do 3 zakrytych kartoników na mapie w wydzielonych kwadratach. Jeden z rozłożonych kartoników musi być prawdziwą miną, pozostałe to miny fałszywe. Informacje o tym, które miny są prawdziwe, a które nie, zna tylko gracz je rozkladający. Każdy gracz posiada limit $M$ min, które może rozłozyć w całej rozgrywce -- ta liczba zależy od rozgrywanego scenariusza i można ją zwiekszyć dzięki jednej z zaawansowanych technologii. Zbudowanie prawdziwej miny kosztuje jeden surowiec $AC$. Gracz może rozłożyć miny w odległości podwojonego zasięgu konstruktora oraz magazynu pod warunkiem posiadania tam surowca $AC$ oraz w kwadracie, w którym nie ma jeszcze ani jednego budynku. Gracz w ramach minowania zamiast wykładania do 3 kartoników może przełożyć jeden wybrany swój kartonik do innego dozwolonego kwadratu bez ponoszenia kosztu surowcowego.}
% 
% \textcolor{orange}{Położenie kartonika z miną w danym kwadracie oznacza jego zaminowanie. Właściciel min może choć nie musi je aktywować, szczególnie gdy inni gracze rozpoczeli budowanie budynków w zaminowanym kwadracie. Po aktywacji miny należy ją odwócić tak żeby każdy gracz ją widział i następuje test jej detonacji przy pomocy kostki K6. Każda mina ma 33.(3)\% szansy na detonacje (chyba że gracz posiada technologie zmieniające ten współczynnik). Test przeprowadza się osobno dla każdego budynku. Jeżeli mina wybuchła to należy zdjąć jej kartonik -- należy jednak rozpatrzeć testy detonacji dla wszystkich budynków w danym kwadracie w tej fazie (może się tak zdarzyć przy gęstej zabudowie). Zdetonowanie miny w poblizu budynku powoduje zniszczenie jednego jego modułu. Jeżeli mina nie została zdetonowana to kartonik  wraz z miną pozostaje w tym kwadracie do kolejnej fazy aktywacji min w następnych turach (pozostaje odkryty jezeli mina była aktywowana). Każdy gracz może położyć jeden kartonik w danym kwadracie, czyli maksymalna liczba kartoników w kwadracie jest równa liczbie graczy biorących udział w rozgrywce.  Po detonacji miny nalezy odłozyć jej kartonik do puli min zużytych.}

%\newpage

\section{Zaawansowane Technologie}

Zaawansowane technologie są kupowane za punkty nauki, które mogą być produkowane przez laboratoria (w zależności od rozgrywanego scenariusza można też przydzielić graczom określoną liczbę technologi, z którymi zaczynają grę -- w sposób losowy lub poprzez intencjinalny wybór). Podstawowy koszt odkrycia technologii to $T$. Gracz wybiera odkrytą technologię w ten sposób, że kładzie przed sobą kartkę z jej nazwą tak, aby inni gracze jej nie widzieli - ten wybór jest tajny. Gracz nie musi ujawniać tej technologii tak długo dopóki nie zdecyduje się na jej użycie -- wtedy musi odwrócić tak kartkę z jej nazwą, aby wszyscy inni gracze mogli ją odczytać. Każda kolejna technologia jest droższa o liczbę punktów równą $Q$ (tzw. poziom przyrostowy), tak jak definiuje to poniższa tabela. Raz odkryta technologia zostaje zapamiętana na cały pozostały czas gry. Gracz musi wybrać nową technologie zaraz po tym, gdy wyprodukował odpowiednią liczbę punktów nauki. Liczby $T$ i $Q$ musza być zdefiniowane na początku rozgywki.
\begin{center}
  \begin{tabular}{| r | c | c | c | c | c | c | c |}
    \hline
    liczba już odkrytych technologii & 0 & 1 & 2 & 3  & ... & $k$ \\
    \hline
    koszt kolejnej technologi & $T$ & $T+Q$ & $T+2Q$ & $T+3Q$ & ... & $T+kQ$ \\
    \hline
  \end{tabular}
\end{center}

\subsection{Spis Zaawansowanych Technologii}
\begin{enumerate}
\item \textbf{Super Komputer $/$ Super Computer}: koszt odkrywania kolejnych technologii jest zredukowany o 2 przyrostowe poziomy  ($T+kQ \to T+(k-2)Q$).
\item \textbf{Rozszerzony Wydział R\&D $/$ Extended R\&D Department}: +50\% szansy na wytworzenie punktu technologi w każdym nieuszkodzonym laboratorium (1 $\to$ 2; rzut kością K6). Dla każdego wytworzonego punktu technologicznego substraty muszą znajdować się w sąsiadującym magazynie.
\item \textbf{Szpiegostwo Przemysłowe $/$ Industrial Spying}: Gracz może skrycie wskazać innego dowolnego gracza przed fazą produkcji punktów nauki. Każdy wyprodukowany punkt nauki trafia też do gracza wykonującego szpiegostwo z prawdopodobieństwem $1/3$.
\item \textbf{Zdalne Odwierty $/$ Remote Drilling}: gracz może podejrzeć jakie złoża przyporządkowane są do danego terenu przed jego odkryciem -- tylko jeden dowolny teren może być podejrzany w danej turze. Dodatkowo w chwili odkrycia przez tego gracza może on wybrać jedno z przyporządkowanych złóż (ich liczbę definiuje scenariusz, ale zwykle są to 2 złoża; $1 \to 2$) -- bez tej technologii wybiera losowo.
\item \textbf{Elastyczna Kolejka $/$ Flexible Order}: gracz może zignorować domyślną kolejność graczy i wybrać jedną z trzech opcji niezależnie w ramach \underline{każdej fazy} tury: pierwszy, ostatni lub bez zmian (w sytuacji gdy następuje kolizja i inni gracze również dysponują tą technologią o wyborze kolejności decyduje rzut kością; gracz który wyrzucił niższy wynik pierwszy decyduje o swojej kolejności, gracz o najwyższym wyniku ostatni podejmuje tę decyzję, a zatem nikt już nie może go przesunąć w kolejce).
\item \textbf{Zaawansowane Kopalnie $/$ Advanced Mining}: +50\% szansy na wydobycie dodatkowego surowca w kopalni, dodatkowy surowiec musi być tego samego typu co pierwszy (1 $\to$ 2; rzut kością K6).
\item \textbf{Zaawansowane Przetwarzanie $/$ Advanced Processing}: +50\% szansy na wytworzenie dodatkowego surowca w nieuszkodzonym mikserze (musi być tego samego typu co pierwszy; 1 $\to$ 2; rzut kością K6; jeżeli test wytworzenia zakończył się sukcesem to koszt substratów obliczne są normalnie dla tego dodatkowego surowca).
\item \textbf{Kompresja$/$Compression}: dodatkowe 2 miejsca w nieuszkodzonym magazynie (4 $\to$ 6). Jeżeli w magazynie jest zgromadzonych 5 lub 5 surowców i zostanie on uszkodzony to należy usunąć 2 wybrane surowce. Dodatkowo magazyn może przyjąć dodatkowy surowiec na turę z kopalń i mikserów (2 $\to$ 3).
\item \textbf{Transformacja Międzysieciowa $/$ Intergrid Transformation}: technologia umożliwa wysyłanie lub odbieranie energii elektrycznej pomiędzy graczami (wystarczy, że tylko jeden gracz ją odkrył, aby przesył energii był możliwy). Wysłanie enrgii innemu graczowi powoduje obniżenie możliwości zasilania własnej infrastruktury.
\item \textbf{Szybka Transmisja $/$ Fast Transsimision}: +50\% szansy na dodatkowy wyrzut w nieuszkodzonym transmiterze przy tej samej szansie wyrzutu (1 $\to$ 2; rzut kością K6). Każdy wytransmtowywany surowiec musi znajdować się w sąsiadującym magazynie.
\item \textbf{System Sensorów $/$ Sensor System}: System sensorów pozwala na optymalizację wynoszenia ładunków na orbitę. Gracz może na cele transmisji traktować trudność wynoszenia towarów na orbitę tak jakby była mniejsza o jeden niż jest w rzeczywistości z prawdopodobieństwem 50\% (rzut kością K6). Gracz może zrezygnować z transmisji, jeżeli ten test zakończył się niepowodzeniem.
\item \textbf{Salwa Pocisków $/$ Missile Salvo}: dodatkowy strzał z nieuszkodzonej wyrzutni rakietowej (2 $\to$ 3; w dalszym ciągu jedna wyrzutnia może zniszczyć tylko jeden moduł jednego budynku chyba, że inna zasada pozwala na rozdzielanie pocisków do różnych celów).
\item \textbf{Wiele Celów $/$ Multiple Targets}: każdej nieuszkodzonej wyrzutni rakietowej można przyporządkować dodtkowy cel, który będzie ostrzeliwany dopiero, jeżeli wyrzutnia zniszczy moduł w pierwszym wybranym budynku (1 $\to$ 2).
\item \textbf{System Przeciw Rakietowy $/$ Anti-Missile System}: Każda nieuszkodzona wyrzutnia rakietowa może podjąć jedną próbę (na turę) anulowania wrogiego trafienia, jeżeli potencjalnie trafiony budynek znajduje się w jej (bazowym) zasięgu. Koszt takiej próby to jeden surowiec $AC$, który musi być do dyspozycji w sąsiadującym magazynie. Prawdopodobieństwo zestrzelenia wrogiej rakiety to $1/2$. System przeciw rakietowy jest automatyczny -- jego użycie oraz listę ochranianych maksymalnie 3 budynków należy zadeklarować skrycie przed wyznaczaniem celów dla każdej wyrzutni osobno. Jeżeli nastąpi deficyt surowców $AC$ to nie można razić już ani celów ani używac przeciwrakiet.
\item \label{gps-tech} \textbf{Satelitarny System Pozycjonowania $/$ Satellite Positioning System}: Prawdopo- dobieństwo trafienia rakietą przez nieuszkodzoną wyrzutnie rakietową w rozszerzonym zasięgu jest  wzrasta do $1/2$ (korekta operatorów podnosi go do $2/3$).
\item \textbf{Pasywne Opancerzenie $/$ Passive Armor}: Do każdego budynku poza elektrownią, można dobudować dodatkowy zapasowy moduł specjalistyczny o koszcie jednego tylko surowca $AB$, który należy zdjąć przy zadaniu uszkodzenia temu budynkowi (czy to w przypadku sabotażu, czy w wyniku ostrzłu rakietowego). Pancerz można odbudować w kolejnych turach, jeżeli został zniszczony. Moduł ten nie generuje mocy w trybie alarmowego zasilania. 
\item \textbf{Penetrator Formowany Wybuchowo $/$ Explosively Formed Projectile}: dzięki tej technologi każdy pocisk wystrzelony z wyrzutni rakietowej może przebić \textbf{Pasywne Opancerzenie} z prawdopodobieństwem $2/3$. Opancerzenie to ulega zniszczeniu i wtedy należy przeprowadzić normalny test zniszczenia modułu przez rakietę (zniszczenie pancerza nie jest wliczane w limit uszkodzeń zadawanych przez jedną wyrzutnie jednemu budynkowi). Jeżeli nie udało się przebić pancerza to tak czy inaczej należy rozpatrzeć skutki zadania obrażeń przez tę rakietę. Udane trafienie powoduje zniszczenie pancerza, ale już nie właściwego modułu.
\item \textbf{Scentralizowany System Dowodzenia $/$ Centralized Command System}: nieuszkodzona wyrzutnia rakietowa może wykorzystywać pole widzenia dowolnego nieuszkodzonego budynku (przy czym określanie pola widzenia dla budynków powinno być określone w rozgrywanym scenariuszu). Gracze moga udostępniać sobie pole widzenia własnych budynków pod warunkiem, że oboje posiadają tę technologię.
\item \textbf{Odzyskiwanie Energi $/$ Energy Recovery}: +20\% wiecej budynków zasilanych jednocześnie przez nieuszkodzony reaktor atomowy, niż wynika to z nominalnej mocy reaktora (zaokrąglając w górę).
\item \textbf{Rekonstrukcja $/$ Construction Recovery}: dodatkowa próba budowy przez konstruktor, gdy poprzednia zakończyła sie niepowodzeniem (1 surowiec i tak jest tracony). W przypadku odzyskiwania surowców, ta technologa pozwala na dodatkową próbę, gdy wcześniejsza zakończyła się niepowodzeniem z powodu uszkodzenia konstruktora.
\item \textbf{Recykling$/$Recycling}: konstruktor może zniszczyć pojedynczy moduł (z wyjątkiem pancerza) dowolnego budynku i odzyskać jeden surowiec $AB$, z którego był zbudowany (2 $\to$ 1). Operacja ta jest wliczana do limitu operacji konstruktora tak jak standardowa operacja budowy. Odzyskany surowiec musi być umieszczony w magazynie w zasięgu tego konstruktora, a następnie może być wykorzystany nawet w tej samej fazie. Jeżeli konstruktor jest uszkodzony to prawdopodobieństwo odzyskania towaru spada do $2/3$ i $1/3$ w zaleznosci od tego czy 1 czy 2 moduły są zniszczone (w porównaniu do bazowej konfiguracji; rzut kością K6). Nieudana próba nie powoduje zniszczenia modułu, ale wyczerpuje limit operacji tego konstruktora, chyba że inna technologia mówi inaczej. Konstruktor nie może odzyskiwać surowców ze swoich własnych modułów.
\item \textbf{Zaawansowana Inżynieria Budowlana $/$ Advance Development}: Konstruktor może przebudować każdy operacyjny budynek (oraz nieoperacyjną elektrownie atomową), tak że zmienia on typ. Nowy budynek ma dokładnie tyle samo nieuszkodzonych modułow co budynek pierwotny. Koszt przebudowy wynosi jeden surowiec $AB$ za każdy moduł nowego budynku. Gracz może zredukować liczbę modułów przy przebudowie -- wtedy każdy moduł różnicy powoduje zmniejszenie kosztu, ale zawsze jest to przynajmniej jeden surowiec. Jeżeli nowym budynkiem jest elektrownia atomowa to jest ona nieoperacyjna do chwili wybudowania jej wszystkich modułów. Nowy budynek jest operacyjny jeżeli pierworny budynek był operacyjny lub gdy posiada wszystkie możliwe moduły.
\item \textbf{Rządowa Dotacja $/$ Government Subsidy}: Gracz może dostawić 3 moduły nowego budynku w nowym miejscu w 5 lub późniejszej turze gry (moduły są zakupione przez rząd wspierający danego gracza i lądują z orbity podczas fazy wyboru technologi -- na końcu tury). Nowy budynek należy umieścić zgodnie z regułami budowy nowych budynków oraz zaczyna on być operacyjny od kolejnej tury, z wyjątkiem elektrowni, którą należy dobudować do końca, żeby zaczeła zasilać infrastrukturę.
\item \textbf{Odział Dywersyjny $/$ Sabotage Unit}: Gracz może sabotować jeden pojedynczy budynek innego gracza, który znajduje się maksymalnie w odległości 10 cm od jakiegokolwiek budynku gracza podejmującego dywersję. Dywersja polega na czasowym wyłączeniu 3 modułów atakowanego budynku na jedną turę. Wyboru budynku należy podjąć przed fazą przydzielania zasilania. Sabotowany budynek nie może być wliczany do budynków generujących zasilanie alarmowe, ale z drugej strony nie wlicza się on też do sumy budynków, które pobierają zasilanie z elektrowni. Jeżeli gracz zdecyduje się na sabotaż tego samego budynku w kolejnych turach, to prawdopodobieństwo udanego sabotażu spada do~$1/3$ -- załoga budynku jest przygotowana.
\item \textbf{Zniszczenie Zapasów $/$ Stock Destruction}: Jeżeli gracz posiada już \textbf{Odział Dywersyjny} i przeprowadza sabotaż magazynu to ta udana akcja dodatkowo kończy się zniszczeniem przynajmiej jednego surowca. Liczbę zniszczonych surowców wyznacza się rzutem kości K6. Jeżeli gracz wyrzucił więcej niż liczba zgromadzonych surowców to wszystkie zgomadzone w magazynie surowce ulegają zniszczeniu. To sabotujący gracz decyduje, które surowce zostały zniszczone.  
\item \textbf{Zniszczenie Modułu $/$ Module Destruction}: Jeżeli gracz posiada już \textbf{Odział Dywersyjny} to ta udana akcja oznacza dodatkowo zniszczenie jednego modułu sabotowanego budynku z prawdopodobieństwem $1/3$. Taka akcja jest bezkosztowa. Jeżeli budynek został zniszczony i nie był wartownią to właściciel traci jeden punkt zwycięstwa.
\item \textbf{Przejęcie Wartowni $/$ Post Capturing}: Jeżeli gracz posiada już \textbf{Odział Dywersyjny} to ta udana akcja dywersji na wartowni przeciwnika oznacza jej przejęcie.
\item \textbf{Przejęcie Ładunku $/$ Glory Takeover}: Jeżeli gracz posiada już \textbf{Odział Dywersyjny} to w ramach jego akcji może przejąć wytransmitowane surowce innego gracza w danej turze przez transmitery znajdujące się w zasięgu 10 cm od dowolnego budynku gracza przejmującego ładunek. Akcję taką można przeprowadzić za pierwszym razem (dla danego gracza okazującego tę technologię) z prawdopodobieństwem 100\% (wystepuje element zaskoczenia). Kolejne próby dokonuje się z prawdopodobieństwem $1/3$. Każdy przejęty surowiec jest zaliczany jako punkty zwycięstwa gracza, który go przejął. 
\item \textbf{Odział Szybkiego Reagowania $/$ Rapid Operational Unit}: Odział szybkiego reagowania, zmiejsza prawdopodobieństwo dywersji do $1/2$ w pierwszej próbie sabotażu danego budynku (po przerwie od dywersji), oraz całkowicie uniemożliwia powtórny sabotaż danego budynku w kolejnej turze. Ta technologia wpełni przeciwdziała sabotażowi nawet, gdy dany gracz jest jednocześnie sabotowany przez wielu graczy. Dodatkowo prawdopodobieństwo przejęcia towaru jest zmiejszane do $1/2$ przy odkryciu technologii przejęcia i do $1/6$ przy kolejnych próbach przejęcia transmitowanego surowca.
\item \textbf{Kampania Marketingowa $/$ Marketing Campaign}: Gracz zyskuje punkt zwycięstwa za każdy moduł operacyjnego budynku zniszczony przeciwnikowi (za wyjątkiem wartowni).
\end{enumerate}

\subsection{Przydzielanie Początkowych Technologii}

Każdy z graczy otrzymuje na początku rozgrywki początkowe technologie, które nie powodują zwiększenia kosztu wymyślania kolejnych technologii. Sposób wyboru i liczbę technologii począt\-kowych ($T$) określa scenariusz.

\section{Kolejność Rozgrywki}

\subsection{Kolejnosć graczy w fazie}

W pierwszej kolejności swoje akcje wykonują gracze o aktualnie najwyższej punktacji zwycięstwa. Chyba, że któryś z graczy posiada odpowiednią technologię zmiany tej kolejności, wtedy należy postępować zgodnie z treścią tej technologii. Każdą kolizje kolejności gracze powinni rozstrzygnąć rzucając kośćmi 2 $\times$ K6 (aż do rozstrzygnięcia).

\subsection{Kolejnosć faz w turze}

\begin{enumerate}
  \setlength{\parskip}{0pt}
  \setlength{\itemsep}{0pt plus 1pt}
\item Ustalanie trudności wynoszenia towarów na orbitę.
\item Rozpatrywanie wpływu czynników naturalnych na infrastrukurę.
\item Możliwość przeprowadzenia zdalnych odwiertów.
\item Wybór sabotowanych budynków / przeprowadzanie sabotażu.
\item Ustalenie zasilania budynków (w tym deklaracja wspomagania zasilania przez innych graczy; wybrane budynki posiadają zasilanie do końca tej tury nawet jeżeli elektrownia zostanie zniszczona).
\item Konstruktor: odzyskiwanie surowców dzięki recyklingowi.
\item Konstruktor: przygotowanie terenu do budowy (ten teren może być przejęty przez innego gracza w wyniku zbudowania przez niego w tym miejscu modułu).
\item Konstruktor: budowanie modułów (jeżeli to pierwszy moduł efektora to nie można go użyć w bieżącej turze; jeżeli to pierwszy moduł kopalni to następuje przyporządkowanie złoża do terenu).
\item Ostrzał rakietowy (w tym zniszczenie modułów).
\item Transmisja surowców na Ziemie.
\item Przejmowanie surowców transmitowanych na Ziemie.
\item Deklaracja szpegostwa przemysłowego.
\item Produkcja punktów technologii.
\item Przemieszczanie surowców pomiędzy magazynami.
\item Wytwarzanie surowców złożonych w mikserach.
\item Wydobywanie surowców prostych w kopalniach.
\item Wybieranie zaawansowanych technologi.
\end{enumerate}

\subsection{Warunki Zwycięstwa}

Punkty zwycięstwa gracz może zdobywać lub tracić w następujący sposób:
\begin{enumerate}
  \setlength{\parskip}{0pt}
  \setlength{\itemsep}{0pt plus 1pt}
\item Gracze zyskują jeden punkt zwycięstwa za każdy wytransferowany towar na Ziemię.
\item Gracze zyskują jeden punkt zwycięstwa za każdą wynalezioną technologię.
\item Gracze zyskują jeden punkt zwycięstwa za każdy zniszczony budynek przeciwnikowi, który był operacyjny (za wyjątkiem wartowni).
\item Gracze tracą jeden punkt zwycięstwa za każdy stracony budynek, które był operacyjny (za wyjątkiem wartowni).
\end{enumerate}
Monitoring punktacji zwycięstwa należy prowadzić jawnie w trakcie rozgrywki. Rozgrywkę wygrywa gracz, który na końcu ma największą ich liczbę.
%Punkty zwycięstwa gracza oblicza się jako różnice wytransmitowanych towarów na Ziemię minus liczbę budynków zniszczonych przez innych graczy (których liczba bazowych modułów była większa niż 1; z wyjątkiem wartowni), które były operacyjne (zniszczenie budynku podczas jego budowy się nie liczy -- z wyjątkiem wyrzutni, która jest operacyjna od pierwszego wybudowanego modułu). 

\end{document}

