\documentclass[11pt,a4paper]{article}
\usepackage[margin=1in,footskip=0.25in]{geometry}
\usepackage[dvipsnames]{xcolor}
\usepackage{amssymb,amsmath}
\usepackage[utf8]{inputenc}
\usepackage[OT4]{fontenc}
\usepackage{graphicx}
\usepackage{multicol}

\begin{document}
\title{Draft: Interplanetary Exploatation Companies}
\author{Krzysztof Czarnecki}
\date{Sierpień, 2023}
\maketitle

\section{Wstęp}

Koniec XXV wieku to okres szybkiego rozwoju górnictwa kosmicznego. Ludzkość intensywnie eksploatuje pobliskie planetoidy oraz planety i ich naturalne satelity, w tym głównie Księzyc, Mars, Wenus, oraz księżyce Jowisza i Saturna. Ważnym źródłem surowców, są także komety przelatujące w poblizu Ziemii. Tu jednak okno czasowe pozwalające na operowanie na ich powieszchnii jest ograniczone do kilku miesięcy ze względu na relatywnie krótki czas przebywania w pobliżu Słońca, które zapewnia na powieszchni tych ciał niebieskich akceptowaną temperaturę dla pracy infrastruktury wydobywczej. Sektor górnictwa kosmicznego zmonopolizowały prywatne przedsiębiorstwa głównie ze Stanów Zjednoczonych, Chin i Indii, ale znana jest również znaczna liczba międzynarodowych konsorcjów tworzonych przez firmy ze wszystkich kontynentów świata poza Antrktydą. Te prywatne korporacje zaczeły organizować się na wzór kolonialnych kopanii handlowych znanych z XVII, XVII i XIX wieku. Mają swoje własne technologie, swoje kosmiczne oddziały militarne, oraz wysoką autonomie od państw macierzystych z racji działanlości poza juryzdykcją prawną tych państw.  

Górnictwo kosmiczne zapewnia ogromne zyski, ale jest też przedsięwzięciem szalenie niebezpiecznym i ze względu na odległość trudnym do kontroli z Ziemii. To powoduje pokusę, aby korporacje kosmiczne stosowały wroge oddziaływania miltarne na konkurentów działających na tych samych złożach surowców. Włączając w to fizyczną eliminację persolelu i infrastruktury wraz z jej siłowym przejmowaniem. Aby, ograniczyć ryzyko takich działań, w roku 2499 wszystkie liczące się w branży kosmicznej państwa na Ziemii podpisały tzw. \textbf{Traktat Pekiński}, który zabrania wynoszenia do przestrzeni kosmicznej bronii ofensywanej rozumianej jako mobilne systemy rażenia. Sygnotariusze tego porozumienia mają obowiązek kontroli działających na ich terytprium korporacji i powinny uniemożliwić wynoszenia takich środków walki do przestrzeni kosmicznej. W rezultacie głównymi systemami militarnymi występującymi na stacjach wydobywczych stały się stacjonarne wyrzutnie rakiet przeznaczone do obrony. Czy jednak inżynierowie i dowódcy tych jednostek będą je wykorzystywać zgodnie z przeznaczeniem? Na to pytanie odpowie czas i zarządy korporacji kosmicznych...
\newpage

\section{Surowce i Processy}

Gracze mogą wydobywać proste surowce oznaczane pojedynczymi literami alfabetu: $A$, $B$ i $C$ dzięki kopalniom. Te proste surowce są wykorzystywane w mikserach jako substraty do produkcji towarów złożonych oznaczonych zwiazkami liter: $AB$, $BC$, oraz $AC$. A zatem kompanie kosmiczne przetwarzają surowce już na powieszchni ciał niebieskich, na których operują. Finalnie, złożone towary są wykorzystywane jako paliwo w tzw. efektorach do produkcji konkretnych środków przydatnych w grze:
\begin{itemize}
  \setlength{\parskip}{0pt}
  \setlength{\itemsep}{0pt plus 1pt}
\item transmisja towarów na ziemie (punkty zwycięstwa),
\item budowanie nowych budynków lub odbudowywanie zniszczonych modułów,
\item wytwarzanie rakiet dla wyrzutni przeznaczonych do niszczenia innych budynków,
\item lub odkrywanie zaawansowanych technologii.
\end{itemize}
\section{Spis Budynków}

\subsection{Ekonomiczne (backend)}
\begin{center}
\begin{tabular}{| c | c | c | c | c |}
  \hline
  \textbf{nazwa PL} & \textbf{nazwa EN} &
  \textbf{moduły} & \textbf{rysunek} & \textbf{zasięg}\\
  \hline
  reaktor nuklearny & nuclear reactor & 8 & \begin{picture}(20, 20)(7,8)
    \put(15,15){\color{blue}\circle{13}}    
    \put(12.5,12.5){\color{blue}\circle{5}}    
    \put(17.5,12.5){\color{blue}\circle{5}}    
    \put(17.5,17.5){\color{blue}\circle{5}}    
    \put(12.5,17.5){\color{blue}\circle{5}}    
  \end{picture} & -- \\  
  \hline
  kopalnia & mine & 1 & \begin{picture}(20, 20)(7,8)
    \put(15,15){\color{blue}\circle{13}}
    \put(15,15){\color{blue}\circle*{4}}
  \end{picture} & 4 cm\\  
  \hline
  magazyn & storage & 2 & \begin{picture}(20, 20)(12,13)
    \put(15,20){\color{blue}\line(1, 0){10}}
    \put(15,15){\color{blue}\line(1, 0){10}}
    \put(15,25){\color{blue}\line(1, 0){10}}
    \put(25,15){\color{blue}\line(0, 1){10}}
    \put(20,15){\color{blue}\line(0, 1){10}}
    \put(15,15){\color{blue}\line(0, 1){10}}
  \end{picture} & 6 cm\\
  \hline
  mikser & mixer & 2 & \begin{picture}(20, 20)(6,8)
    \put(15,15){\color{blue}\circle{13}}
    \put(10,10){\color{blue}\line(1, 1){10}}
    \put(10,20){\color{blue}\line(1, -1){10}}
  \end{picture} & 4 cm\\
  \hline
\end{tabular}
\end{center}

\subsection{Efektory (frontend)}
\begin{center}
  \begin{tabular}{| c | c | c | c | c | c |}
    \hline
    \textbf{nazwa PL} & \textbf{nazwa EN} & \textbf{moduły} &
    \textbf{rysunek} & \textbf{zasięg} & \textbf{substraty} \\
    \hline
    laboratorium & laboratory & 2 & \begin{picture}(20, 20)(14,2)
      \put(25,10){\color{blue}\circle{7}}
      \put(15,10){\color{blue}\line(1, 1){10}}
      \put(15,10){\color{blue}\line(1, -1){10}}
      \put(35,10){\color{blue}\line(-1, 1){10}}
      \put(35,10){\color{blue}\line(-1, -1){10}}
    \end{picture} & -- & $AB + BC + AC$\\
    \hline
    transmiter & transsmiter & 2 & \begin{picture}(20, 20)(14,-2)
      \put(25,5){\color{blue}\circle{13}}
      \put(22,0){\color{blue}\rotatebox{90}{$\gg$}}
    \end{picture} & -- & $BC$\\    
    \hline
    konstruktor & developer & 3 & \begin{picture}(20, 20)(14,-2)
      \put(25,5){\color{blue}\circle{13}}
      \put(21.5,10){\color{blue}\rotatebox{-90}{$\gg$}}
    \end{picture} & 6cm & $AB$\\    
    \hline
    wyrzutnia rakietowa & launcher & 3 & \begin{picture}(20, 20)(5,8)
      \put(15,15){\color{blue}\circle{13}}
      \put(15,6){\color{blue}\line(0, 1){18}}
      \put(6,15){\color{blue}\line(1, 0){18}}
    \end{picture} & 10 cm & $AC$\\
    \hline
  \end{tabular}
\end{center}

\subsection{Reaktor Nuklearny $/$ Nuclear Reactor}

Pojedynczy reaktor nuklearny jest małą modułową elektrownią elektryczną -- może zasilać jednocześnie $X$ budynków (sam reaktor atomowy również musi być zasilany, a zatem wchodzi w skład tej liczby). W przypadku przekroczenia wspomnianej liczby na skutek, czy to zniszczenia reaktora, czy zbudowania zbyt wielu budynków gracz musi wybrać, które budynki są zasilane a które nie. Niezasilane budynki pozostają nieaktywne do chwili powtórnego przywrócenia zasilania -- można zmieniać konfigurację zasilanych budynków w każdej turze rozgrywki. Reaktor atomowy zaczyna produkować energię elektryczną dopiero wtedy, gdy zbydowano wszystkie jego moduły. Działający reaktor przestaje zasilać dopiero wtedy, gdy zostały zniszczone wszystkie jego moduły. Wystarczy choćby jeden sprawny moduł, aby zasilać nominalną liczbę budynków $X$. Zniszczenie wszystkich reaktorów atomowych, co oznacza wyłączenie możliwości zasilania dla wszystkich budynków, powoduje natychmiastowe wyeliminowanie gracza z rozgrywki -- jego punkty za dotychczasowo wytransmitowane towary są dzielone przez 2. Każdy gracz może wspomagać dowolnego innego gracza w zasilaniu jego infrastruktury -- w takim wariancie gracz może dalej uczestniczyć w rozgrywce. Należy jednak pamietać, że zasilanie budynków innego gracza powoduje zmiejszenie możlwości zasilania swoich własnych budynków. Gracz udzielający pomocy decyduje ile budynków otrzymuje zasilanie, natomiast to właściciel infrastruktury zawsze decyduje, które budynki są zasilane, a które nie. Reaktor nuklearny powinien znajdować się w startowym zestawie budynków, chyba, że rozgrywany scenariusz mówi inaczej.

\section{Zaawansowane Technologie}

Zaawansowane technologie są kupowane za punkty nauki, które mogą być produkowane przez laboratoria. Podstawowy koszt odkrycia technologii to $T$. Gracz wybiera odkrytą technologię w ten sposób, że kładzie przed sobą kartkę z jej nazwą tak, aby inni gracze jej nie widzieli - ten wybór jest tajny. Gracz nie musi ujawniać tej technologii tak długo dopóki nie zdecyduje się na jej użycie -- wtedy musi odwrócić tak kartkę z jej nazwą, aby wszyscy inni gracze mogli ją odczytać. Każda kolejna technologia jest droższa o liczbę punktów równą $Q$, tak jak definiuje to poniższa tabela. Raz odkryta technologia zostaje zapamiętana na cały pozostały czas gry. Gracz musi odkryć nową technologie zaraz po tym, gdy wynalazł odpowiednią liczbę punktów nauki. Liczby $T$ i $Q$ musza być zdefiniowane na początku rozgywki.
\begin{center}
  \begin{tabular}{| r | c | c | c | c | c | c | c |}
    \hline
    liczba już odkrytych technologii & 0 & 1 & 2 & 3  & ... & $k$ \\
    \hline
    koszt kolejnej technologi & $T$ & $T+Q$ & $T+2Q$ & $T+3Q$ & ... & $T+kQ$ \\
    \hline
  \end{tabular}
\end{center}

\subsection{Spis Zaawansowanych Technologii}
\begin{enumerate}
\item \textbf{Advanced Mining}: +50\% szansy na wydobycie dodatkowego surowca w kopalni, dodatkowy surowiec musi być tego samego typu co pierwszy (1 $\to$ 2; rzut kością K6).
\item \textbf{Advanced Processing}: +50\% szansy na wytworzenie dodatkowego surowca w nieuszkodzonym mikserze przy tym samym koszcie (1 $\to$ 2; rzut kością K6).
\item \textbf{Effective Compression}: dodatkowe miejsce w nieuszkodzonym magazynie (4 $\to$ 5).
\item \textbf{Fast Transsimision}: +50\% szansy na dodatkowy wyrzut w nieuszkodzonym transmiterze przy tej samej szansie wyrzutu (1 $\to$ 2; rzut kością K6).
\item \textbf{Construction Recovery}: dodatkowa próba budowy przez dewelopera, gdy poprzednia zakonczyła sie niepowodzeniem (1 surowiec i tak jest tracony).
\item \textbf{Missile Salvo}: dodatkowy strzał z nieuszkodzonej wyrzutni rakietowej (2 $\to$ 3; w dalszym ciągu jeden launcher może zniszczyć tylko jedną część jednego budynku).
\item \textbf{Energy Recovery}: +20\% wiecej budynków zasilanych jednocześnie przez nieuszkodzony reaktor atomowy niż wynika to z nominalnej mocy reaktora (zaokrąglając w górę).
\end{enumerate}


\end{document}

